%英文摘要,自行编辑内容




\chapter{ABSTRACT}
\xiaosi
% Point cloud semantic segmentation stands as a core task in 3D scene understanding, aiming to accurately map each point in a scene to predefined semantic categories. This technology finds widespread applications in autonomous driving, intelligent robotics, virtual reality, and augmented reality. The rapid development of deep learning has enabled efficient processing of large-scale point cloud data and facilitated the emergence of numerous high-performance open-source models. However, despite advancements in data acquisition, fully supervised training remains constrained by the labor-intensive nature of acquiring point-wise annotations, which are critical yet prohibitively costly to obtain at scale. Furthermore, models trained on source datasets often suffer significant performance degradation when deployed on new target datasets due to domain gaps caused by variations in data acquisition devices and environmental conditions. Domain adaptation has emerged as a pivotal strategy to mitigate domain shifts and reduce annotation dependency. Current approaches predominantly adopt unsupervised or semi-supervised paradigms, which reduce annotation costs but inevitably sacrifice segmentation accuracy. In contrast, active domain adaptation offers a cost-effective balance between human annotation efforts and model performance. Nevertheless, dedicated active query strategies for 3D point cloud semantic segmentation remain underdeveloped, and existing methods fail to effectively integrate actively selected target-domain points with labeled source-domain data.
% To address these challenges, this thesis systematically reviews research progress and methodologies in active learning and domain adaptation for visual semantic segmentation, highlighting the limitations of existing approaches. We further propose two innovative solutions for 3D point cloud semantic segmentation, and the following methods are established:
Point cloud semantic segmentation is the core task of 3D scene understanding, and its goal is to accurately map each point in the scene to a preset semantic class, which is widely used in the fields of unmanned driving, intelligent robotics, and virtual and augmented reality. 
% With the rapid development of deep learning technology, efficient processing of large-scale point cloud data has become possible and has given rise to a large number of excellent open-source models. 
With the rapid development of deep learning technology, efficient processing of large-scale point cloud data becomes possible and gives rise to a large number of excellent open-source models.
% However, although point cloud data acquisition has become more convenient, fully supervised training requiring point-by-point labeling is still difficult to obtain, because point-by-point labeling of large-scale point clouds is a time-consuming and labor-intensive task. 
However, although point cloud data acquisition becomes more convenient, fully supervised training requiring point-by-point labeling remains difficult to obtain, because point-by-point labeling of large-scale point clouds is a time-consuming and labor-intensive task.
In addition, due to the differences in different acquisition devices and scenarios, there are often significant domain gaps when directly applying models trained on source datasets to new target datasets, resulting in performance degradation. As an important strategy to solve the domain gap problem and reduce the dependence on annotation, domain adaptation currently mainly adopts unsupervised or semi-supervised modes, which saves the annotation cost but often sacrifices part of the model performance. In contrast, active domain adaptation provides a more cost-effective way to balance between labor cost and segmentation accuracy. 
% However, no specialized active query strategy has yet been proposed for active domain adaptation for semantic segmentation of 3D point clouds, and there is a lack of schemes to effectively combine the actively selected target domain points with the annotated source domain data. 
However, no specialized active query strategy has been proposed for active domain adaptation for semantic segmentation of 3D point clouds, and there remains a lack of schemes to effectively combine the actively selected target domain points with the annotated source domain data.
To address the above issues, this thesis categorizes and summarizes the research progress and methods of active learning and domain adaptation in visual semantic segmentation, and describes the advantages and disadvantages of the existing methods. In addition, effective domain-adaptive active learning methods for semantic segmentation of 3D point clouds as well as active mixing methods that fully utilize the potential of both domains are further proposed as shown below:

% 1. Domain discrepancy aware active learning for cross-domain segmentation. The main contributions are as follows: 1) We introduce a prototype-guided query strategy that computes the cosine distance between target point cloud features and source domain class prototypes. By integrating this distance with predictive entropy, we construct a joint evaluation metric that effectively selects high-transfer-value points in cross-domain scenarios. 2) We pioneer the combination of active learning with a mixing strategy in the context of domain adaptation for point cloud segmentation, blending actively selected target points with source domain data to create robust intermediate domain samples, thereby further reducing the domain gap.
1. A Domain discrepancy aware active learning for cross-domain segmentation, its main contributions are as follows: 
% 1. A domain discrepancy-aware active learning method for cross-domain segmentation is proposed, with the following contributions:
1) Propose a prototype-guided domain discrepancy awareness query strategy, by calculating the cosine distance between the point cloud features of the target domain and the prototype centers of the source domain classes, and constructing a joint evaluation index of “feature distance-predictive entropy” to achieve the selection of points with high migration value in cross-domain scenarios. 
% 2) For the first time, this thesis combined the active learning with the mixing method and used it for the domain adaptation task of point cloud semantic segmentation, and constructed a robust intermediate domain data to further reduce the domain gap by mixing the target points selected by active learning with source domain points.
2) For the first time, this thesis combines active learning with the mixing method and uses it for the domain adaptation task of point cloud semantic segmentation, constructing robust intermediate domain data to further reduce the domain gap by mixing the target points selected by active learning with source domain points.

% 2. Active mixing method for cross-domain semantic segmentation. The main contributions are as follows: 1) We propose a source-target quantity balancing algorithm that ensures an equal number of annotated points from both domains are mixed, enabling the model to learn a more balanced representation and mitigating cumulative domain shift during training. 2) Building on this, we further introduce a category-balanced active mixing algorithm to address class imbalance in the mixed intermediate domain, which enhances overall model performance.
% 2. Active mixing method for cross-domain semantic segmentation, its main contributions are as follows: 1) Propose a source-target amount balance algorithm with the source-target amount balance algorithm, mixing the source and target domain points with the same number of labeling, so that the model can learn a more balanced knowledge of the two domains, and solving the problem of accumulating domain bias in the process of model learning. 
2. Active mixing method for cross-domain semantic segmentation, its main contributions are as follows:
 1) Explore the combination of active learning and hybrid methods in depth, and propose a source-target amount balance algorithm that mixes source and target domain points with an equal number of labels, so that the model learns a more balanced representation of the two domains and resolves the issue of accumulating domain bias during model learning.
2) Propose the class-balanced active mixing algorithm on the basis of the source-target quantity balance module, solving the mixing intermediate domain class imbalance problem and further improve the performance of the model.

% Extensive experiments on both synthetic-to-real and real-to-real cross-domain tasks demonstrate that our proposed methods outperform state-of-the-art domain adaptation approaches, achieving superior performance even under scenarios with extremely limited annotations.
% The method proposed in this thesis has been extensively experimented and validated on synthetic-to-real and real-to-real cross-domain tasks. The experimental and visualization results fully demonstrate the superiority of the method by comparing with similar existing excellent domain adaptive methods, even achieving results beyond full supervision in very few labeled cases.
The method proposed in this thesis is extensively experimented on and validated in both synthetic-to-real and real-to-real cross-domain tasks. The experimental and visualization results fully demonstrate the superiority of the method when compared with similar state-of-the-art domain adaptation methods, even achieving performance beyond that of full supervision under very few labeled cases.\\
% Dissertation /Thesis is postgraduate’s main academic performance to display her/his works of scientific research, which shows the author’s new invention, new theory or new opinion in her/his research. It is the crucial document for the graduate students to apply for degree, and it is also the important scientific research literature and the valuable wealth of society.
% In order to further standardize the format of dissertation/thesis writing and improve graduate dissertation/thesis quality, this temolate is formulated with reference to the national standard "Rules for Dissertation Writing" (GB/T 7713.1-2006) and the reality of CQUPT.

% \noindent\textbf{Keywords:}Point Cloud Processing, 3D Vision, Semantic Segmentation,  Domain Adaptation, Active Learning, Mixing Methods
\noindent\textbf{Keywords:} 
\begin{minipage}[t]{0.85\linewidth}
	Point Cloud Processing, 3D Vision, Semantic Segmentation,  Domain Adaptation, Active Learning, Mixing Methods
\end{minipage}

\clearpage