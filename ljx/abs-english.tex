%英文摘要,自行编辑内容




\chapter{ABSTRACT}
\xiaosi
Point cloud semantic segmentation stands as a core task in 3D scene understanding, aiming to accurately map each point in a scene to predefined semantic categories. This technology finds widespread applications in autonomous driving, intelligent robotics, virtual reality, and augmented reality. The rapid development of deep learning has enabled efficient processing of large-scale point cloud data and facilitated the emergence of numerous high-performance open-source models. However, despite advancements in data acquisition, fully supervised training remains constrained by the labor-intensive nature of acquiring point-wise annotations, which are critical yet prohibitively costly to obtain at scale. Furthermore, models trained on source datasets often suffer significant performance degradation when deployed on new target datasets due to domain gaps caused by variations in data acquisition devices and environmental conditions. Domain adaptation has emerged as a pivotal strategy to mitigate domain shifts and reduce annotation dependency. Current approaches predominantly adopt unsupervised or semi-supervised paradigms, which reduce annotation costs but inevitably sacrifice segmentation accuracy. In contrast, active domain adaptation offers a cost-effective balance between human annotation efforts and model performance. Nevertheless, dedicated active query strategies for 3D point cloud semantic segmentation remain underdeveloped, and existing methods fail to effectively integrate actively selected target-domain points with labeled source-domain data.
To address these challenges, this paper systematically reviews research progress and methodologies in active learning and domain adaptation for visual semantic segmentation, highlighting the limitations of existing approaches. We further propose two innovative solutions for 3D point cloud semantic segmentation, and the following methods are established:

1. Domain discrepancy aware active learning for cross-domain segmentation. The main contributions are as follows: 1) We introduce a prototype-guided query strategy that computes the cosine distance between target point cloud features and source domain class prototypes. By integrating this distance with predictive entropy, we construct a joint evaluation metric that effectively selects high-transfer-value points in cross-domain scenarios. 2) We pioneer the combination of active learning with a mixing strategy in the context of domain adaptation for point cloud segmentation, blending actively selected target points with source domain data to create robust intermediate domain samples, thereby further reducing the domain gap.

2. Active mixing method for cross-domain semantic segmentation. The main contributions are as follows: 1) We propose a source-target quantity balancing algorithm that ensures an equal number of annotated points from both domains are mixed, enabling the model to learn a more balanced representation and mitigating cumulative domain shift during training. 2) Building on this, we further introduce a category-balanced active mixing algorithm to address class imbalance in the mixed intermediate domain, which enhances overall model performance.

Extensive experiments on both synthetic-to-real and real-to-real cross-domain tasks demonstrate that our proposed methods outperform state-of-the-art domain adaptation approaches, achieving superior performance even under scenarios with extremely limited annotations.
% Dissertation /Thesis is postgraduate’s main academic performance to display her/his works of scientific research, which shows the author’s new invention, new theory or new opinion in her/his research. It is the crucial document for the graduate students to apply for degree, and it is also the important scientific research literature and the valuable wealth of society.

% In order to further standardize the format of dissertation/thesis writing and improve graduate dissertation/thesis quality, this temolate is formulated with reference to the national standard "Rules for Dissertation Writing" (GB/T 7713.1-2006) and the reality of CQUPT.
\noindent\textbf{Keywords:}Point Cloud Processing, 3D Vision, Point Cloud Semantic Segmentation,  Domain Adaptation, Active Learning, Mixing Methods

\clearpage