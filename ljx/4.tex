\chapter{基于点云语义分割域适应的主动混合策略}
\thispagestyle{others}
\pagestyle{others}
\xiaosi

    \section{本章引言}
    本章主要介绍基于点云语义分割域适应的主动混合策略。该方法通过将基于原型指导的主动学习方法和本节所提出的主动混合策略进行深度结合,进一步提升了主动域适应方法的结果。接下来,下文将分别从方法提出的研究动机及主要贡献、方法的具体组成和实现细节,以及实验分析评估等方面,对所提出的方法进行全面阐述。 

    \section{研究动机及贡献}
    % 首先说研究背景和问题,再说一些解决的方法,然后提一下前者怎么做的,后者怎么做的
    % 第二就是说一些文章,他们是怎么使用这些方法解决问题的
    % 第三就是提上述文章方法的不足,
    % 核心点是:主动域适应中主动学习与mixing的结合不够深入,仍然有很大的探索空间。
    % 背景现状(标注问题)-> 一些现存的方法 -> 域适应中和主动学习以及mixing的方法 -> mixing方法构建中间域,举例一些方法,然后这些方法在mixing的不足,尽管如此,但是mixing与主动学习在域适应的结合是从没有过,或者说mixing与主动学习的结合从未有过探索,但是它们却有着巨大的探索价值 -> 我们从上一章节看到了mixing带来的提升,但是基于AL的mixing还有更大的探索空间,对于这两个的结合的探索仍然缺乏。
    % 一些研究者将mixing用于半监督域适应中
    % 主动域适应的问题,以及mixing却可以为之互补。
    % 域适应中有主动域适应和一些基于无监督或者半监督的域适应。
    % 然而在域适应中主动学习方法和mixing
    % 提及一些主动域适应的方法以及已无监督域适应的一些mixing方法
    % 在。。中,。。。使用mixing。。。的提升了模型性能,域适应中。。。使用mixing提升了性能,在。。。中通过主动域适应得到了很好的解决,mixing和主动学习分别为标注问题默默贡献着。或者直接说主动域适应的问题,然后引出一些mixing方法的好处,可以完美的进行结合。
    % 科技的进步使得点云雷达数据的获取变得更加容易,深度学习技术的发展则使得点云语义分割任务得到了快速进步,一些优秀的模型接踵而至,并取得了令人瞩目的结果。但是,这些优秀模型都是基于全监督模式下的,需要点云样本进行逐点的标注,而点云标注则是一项耗费人力物力的艰巨任务。为了缓解这以问题,一些优秀的方法被提出,而域适应就是其中之一。在无监督和半监督域适应中,一些方法独出心裁,使用mixing的方法构造中间域来学习特征的不变性,进而解决域偏差问题。polarmix[]。提出将不同的雷达线速进裁剪并混合。CoSMix[]。将指定语义类别的源域点和伪标签置信度高于阈值的目标点进行过交换混合。而主动域适应[HPML、CLUE、SDM]则是通过选择对缩减域偏差帮助最大的目标点,并进行标注后以提升模型的性能。
    科技的进步使得获取点云雷达数据变得更加容易,而深度学习技术的发展则推动了点云语义分割任务的迅速提升,涌现出一系列优秀模型并取得了显著成果。然而,这些模型大多基于全监督模式,需要对点云样本进行逐点标注,这是一项耗费大量人力物力的艰巨任务。为缓解这一问题,研究者们提出了多种方法,而域适应就是其中一种有效的策略。在无监督和半监督域适应中,一些方法巧妙地使用混合策略构造中间域,以学习特征不变性,进而解决域偏差问题。Polarmix\upcite{xiao2022polarmix}方法通过裁剪并混合不同雷达线束实现源和目标的混合。CoSMix\upcite{saltori2022cosmix}方法则交换并混合源域中指定语义类别的点与目标域中伪标签置信度高于阈值的点。此外,主动域适应方法\upcite{CLUE,HPML,SDM}通过主动选择对减小域偏差最有帮助的目标点进行标注,从而提升模型性能。 

    尽管上述算法各自以不同方式缓解了标注问题,提升了分割模型的性能,但在域适应中,主动学习与混合策略的结合尚未得到深入研究。无论是在图像还是三维点云领域,混合策略和主动学习通常独立应用,分别为减少标注需求做出贡献。然而,现有的混合方法若直接应用于主动域适应,可能存在以下问题:1)数量不均衡导致的域偏差问题。现有的混合方法多基于同一数据集或大量伪标签,特点是场景连续且点数丰富。在域适应中,源域和目标域存在域偏移,主动学习预算下选择的点数量有限。若直接应用这些混合方法,可能导致模型过度学习源域信息,阻碍对目标域信息的提取,最终导致次优结果。因此,在主动域适应中,保持源域和目标域信息的平衡可能比场景连续性更为重要。主动学习选点导致的语义类别不平衡问题。大多数主动学习选点策略基于不确定性。在实际标注前,无法确定所选子集的类别分布是否均衡。尽管一些方法通过伪标签预判类别,但这基于模型不可靠的预测,在域适应任务中可能不适用。此外,现有的混合方法未考虑逐点的语义类别平衡问题。因此,如何使主动学习选择的目标子集与源数据混合后的训练子集在类别分布上尽可能平衡,仍是一个普遍存在的问题。
    针对上述问题,本章进一步探讨了主动学习方法与混合策略在域适应任务中的深度结合。在该方法中,首先提出了基于原型指导的主动学习与混合策略相结合的基础框架。在此基础上,提出了面向点云语义分割域适应的源-目标数量平衡算法和类别平衡主动混合算法。源-目标数量平衡算法旨在解决数量不均衡导致的域偏差问题,类别平衡主动混合算法用于解决主动学习选点导致的语义类别不平衡问题。这两个算法有效地实现了主动学习与Mixing方法的深度结合,进一步提升了模型的分割性能。
    % 虽然上述算法都通过自己的方式解决了标注问题,使得分割模型有了一定的提升,但是对于域适应中主动学习和mixing的结合并未有人进行过深入的探索,无论是在图像还是三维点云领域,mixing和主动学习独立且分别为缓解标注问题默默贡献着。而现存的mixing方法如果直接运用到主动域适应中可能存在以下问题:1)数量不均衡而导致域偏差问题,由于之间的mixing方法更多的是基于同一数据集或者大量的伪标签的前提下的,因此这些方法有着共同的特点:场景连续且点数更多。而在域适应中,目标域和源域之间存在域偏移问题,并且主动学习预算选择的点非常少,因此如果直接将这些mixing方法运用到主动域适应中,可能会导致模型因学习到更多源域信息而阻碍对目标域信息的提取,得到次优的模型结果。因此在主动域适中,场景的连续可能并不是那么重要,相同的源-目标域信息可能才是进一步提升模型性能的最佳混合方法。2)主动学习选点导致的语义类别不平衡问题。目前为止,大多数主动学习的选点策略都是基于不确定性的,在真实标注之前,我们无法得知最终选择的子集的类别分布是否均衡,虽然有一些方法通过伪标签的方式提前预判选择的类别,但这仍然是基于模型的不可靠预测的伪标签,在域适应任务中可能并不适用,现有的mixing未考虑逐点即语义类别平衡的问题,因此存在一个普遍的问题,
    % 如何使得主动学习选择后的目标子集与源数据混合后的训练子集是类别分布尽可能得平衡。对于上述问题,本章节进一步探索了用于域适应任务的主动学习方法与mxing深度结合。在该方法中,首先将提出的基于原型指导的主动学习与Mixing结合形成了基础结合框架,并基于该框架提出了面向点云语义分割域适应的源-目标数量平衡算法和类别平衡主动混合算法。源-目标数量平衡算法解决数量不均衡而导致域偏差问题,类别平衡主动混合算法用于解决主动学习选点导致的语义类别不平衡问题。两个算法有效的完成了主动学习和mixing方法的深入结合,并进一步提高了模型分割的性能。
    
    本章研究的主要贡献如下:

    1)提出了源-目标数量平衡算法,通过选择与目标域样本中标注数量相同的源点进行混合构建中间域,解决了因数量不平衡导致的域偏差问题,进一步提高了混合中间域的有效性。

    2)提出了类别平衡主动混合算法,利用在源-目标数量平衡算法,得到多个候选源点混合子集,计算并选择与标注的目标域混合后类别分布最为均衡的子集进行mixing,借助源域数据实现主动选点的类别平衡,进一步提升了主动学习的有效性。

    3)根据提出的两个算法模块实现了主动学习与mixing的深入结合,得到了深度融合主动混合的点云语义分割域适应框架,分割表现领先于所有现存域适应方法。

    % 为了更进一步提升在主动域适应中主动学习方法与Mixing的进一步结合,提出了基于点云语义分割域适应的主动混合策略。在该方法中,首先将提出的基于原型指导的主动学习与Mixing结合形成了基础结合框架,并基于该框架提出了面向点云语义分割域适应的源-目标数量平衡算法和类别平衡主动混合算法,两个算法有效的适配了主动域适应中的主动学习,减小了域差异和主动选择中类别不平衡则一普遍存在的问题,并进一步提高了模型分割的性能。

    \section{基于点云语义分割域适应的主动混合策略}
    在第三章中,我们设计了基于原型的主动学习方法,并初步尝试结合混合方法构建强健的中间域数据,以缩减域偏差,提升模型性能。然而,混合方法与主动学习的结合仍有广阔的探索空间。如何根据域偏差以及主动学习的特点来有效结合混合方法,仍是一个值得深入研究的问题。

    为了深入探讨混合(Mixing)方法与主动学习在域适应任务中的深度融合,本章在第三章提出的总体框架基础上进行了改进,改进后的框架如图 x-x 所示。该算法的基本流程主要由四个主要模块构成:\ding{172}源域原型构建模块;\ding{173}源原型引导的数据选择模块(SPAL);\ding{174}源-目标数量平衡模块(STNB);\ding{175}类别平衡主动混合模块(CBAM)。为进一步实现两者的高效协同作用,从而获得更优的分割表现,本章对第三章的动态中间域构建模块进行了改进,分为两个模块:一是源-目标数量平衡模块,选择类别数量大于预设定值的源域帧,并从中筛选与所匹配的目标域帧标注点数平衡的候选子集;二是类别平衡主动混合模块,计算每个候选子集与目标域帧标注点混合后的类别分布熵值,选择熵值最大的候选源域子集,以构建类别相对平衡的中间域数据。

    通过上述改进,旨在实现混合方法与主动学习在域适应任务中的深度融合,进一步提升模型的分割性能。 
    % 为了探索mixing方法与主动学习在域适应任务上的深度结合,本章方法延续第三章的总体框架进行改进,改进后的框架如图 x-x 所示。算法基本流程主要由四个主要模块构成:\ding{172}源域原型构建模块,\ding{173}源原型引导的数据选择模块(SPAL),\ding{174}源-目标数量平衡模块以及\ding{175}类别平衡主动混合模块。在第三章中,我们设计了基于原型的主动学习方法,并初步探索结合Mixing方法构建出强壮的中间域数据,进一步缩减了域偏差,提升了模型的性能。然而对与mixing与主动学习的结合仍然有很大的探索空间。如何根据域偏差以及主动学习的特点来结合mixing仍然时一个可以探索的巨大问题。为了进一步形成两者间的高效促进作用从而获得更好的分割表现,本章方法在第三章的动态中间域模块进行了改进,分为了两个模块。一个是源-目标数量平衡模块:选择大于预设定类别数量的源域帧,并从中筛选与所匹配的目标域帧标注点数平衡的候选子集。另一个则是类别平衡主动混合模块:计算每个候选子集与目标域帧标注点混合后的类别分布熵值,选择值最大的候选模块构建类别相对平衡的中间域数据。
    
    % 我们初步探索结合了Mixing与主动学习并应用到语义分割域适应当中,并得到了可喜的效果

    % 应该是介绍一下总体的框架,然后说一些模块的组成,接着再说一下流程以及各模块在流程中的作用。
    \subsection{源-目标数量平衡算法}
    % 说一下问题及原因 - 引出我们的方法,接着对我们的方法的实现进行介绍(600个字左右就行)
    % 就是说明这个算法模块是怎么搞的

    前一章节的实验结果证明了主动学习与Mixing方法的结合取得了显著的效果。然而这只是对于两种方法结合的初步探索,对于他们的深度结合仍然有很大的探索空间。在主动域适应中,主动学习标注点的数量极少,又加之域间隙的存在,使得之前的连续场景、或者基于伪标签的Mixing方法与主动学习结合并不能发挥其最大的效果。因此,本章的源-目标平衡算法旨在数量层名平衡混合的源域点和标注的目标点,对主动学习与Mixing的深入结合进行探索。
    
    在每一轮主动学习迭代中,将最新标注的目标点加入到已标注数据集中,更新数据集$\mathcal{T}^{al}_r$,更新公式如\eqref{eq:al_target_update}所示:
    \begin{equation}
        \label{eq:al_target_update}
        \mathcal{T}^{al}_r = \mathcal{T}^{al}_{r-1} \cup \mathcal{T}^{al}_r 
    \end{equation}
    其中,$\mathcal{T}^{al}_r$是当前主动学第$r$轮的主动标注最新数据集,而$\mathcal{T}^{al}_{r-1}$是上一轮即第$r-1$轮主动标注的数据集,其中$r$从1开始计算,并且当$r=0$时,$\mathcal{T}^{al}_0$为空集$\emptyset$。
    在主动学习阶段结束后,接着就是将已标注的最新的目标域数据与源数据混合。
    选择一帧带有标注的目标域数据,然后再随机匹配一帧源域数据,
    % 当然源域帧的筛选是有条件的,只有当帧中的类别的数量大于我们设定的阈值$\tau$时,这个源数据帧才有资格与当前的目标域帧进行混合,
    为了保证数据的质量,源域数据的筛选是有条件的,即仅当某一帧点云的类别数量大于预设的阈值$\tau$时,该帧数据才有资格与当前目标域数据进行混合,筛选过程如\eqref{eq:filter_source}所示,其中$\mathbf{S}_i = \{\mathbf{X}^S_i,\mathbf{Y}^S_i\}$,表示源域中的一帧点云数据。
    \begin{equation}
        \label{eq:filter_source}
        % \mathcal{S}_{mix} 
        \mathcal{M}_{mix}= \{\mathbf{S}_i | unique(\mathbf{Y}^S_i)> \tau\}, \quad \mathbf{Y}^S_i \in \mathbf{S}_i
    \end{equation}
    接着,我们从$\mathcal{M}_{mix}$中随机筛选一个源域帧$\mathbf{M}_i \in \mathcal{M}_{mix}$与目标域帧$\mathbf{p}_i$进行匹配,在匹配成功后,我们将从匹配的源域帧中,候选$m$个混合子集,这些候选子集中的点数与匹配的目标域中主动标注的点数量相同,候选点的公式如下所示:
    % 首先,对于进行混合选择的源域数据进行过滤筛选,
    \begin{equation}
        \label{eq:mix_subset}
        \mathbf{Q} = \{\mathbf{q}_1,...,\mathbf{q}_m\}, 
        \quad
        \|\mathbf{q}_i\| = \|\mathbf{p}_i\|,
        \quad
        \mathbf{q}_i \subset \mathbf{Q},
        \quad
        \mathbf{p}_i \subset \mathcal{T}^{al}_r
    \end{equation}
    式中,$\mathbf{Q} \subset \mathbf{M}_i$为$m$个源域候选混合子集的集合,$\mathbf{q}_i$带表第$i$个候选子集,$\mathbf{p}_i$代表一个目标域点云帧中已标注点的集合。$\|\mathbf{q}_i\|$和$\|\mathbf{p}_i\|$分别代表候选子集$\mathbf{q}_i$和目标点云已标注点$\mathbf{p}_i$的数量。

    \subsection{类别平衡主动混合算法}  %使用select的进行写吧,看一下那篇文章
    % 同上,可以参考SELECT,但是一定要注意的是把体素换成是我们的点云帧才行(1000个字左右)
    % 主动学习普遍存在的一个问题是类别不平衡问题,所选择的点是无法提前预知的,所有的策略都是基于模型进行推测的,因此目标域中每一帧点云各类别所含点的数量差异显著。而标注后的类别分布的结果后验于主动学习选择的结果,为了解决这个问题,在域适应任务中源域中拥有大量且标注的语义类别点,本章的方法就是选择一些语义分布更加均衡的源域点进行混合,这样既可以学习到域不变特征,又可以缓解类别不平衡的问题。
    主动学习普遍存在类别不平衡的问题。由于所选取的点无法提前预知,所有策略均基于模型推测,标注后的类别分布结果又依赖于主动学习的选择,这会导致目标域中每一帧点云各类别所含标注点数存在显著差异。而在域适应任务中,源域通常拥有大量带标注的语义类别点。因此,为解决这一问题,本章方法通过选择可以使得混合后类别分布更平衡的源域点,并使之与目标域标注点进行混合形成中间域数据并用于模型微调,这不仅有助于学习域不变特征,也能有效缓解类别不平衡的问题。

    基于源-目标数量平衡模块所得到的$m$个候选子集,从$\mathbf{Q}$中选取一个候选子集$\mathbf{q}_i$,使得在所有被选中候选子集中的类别分布尽可能平衡。为此将根据混合后的点云集真实标签分布计算的熵作为选择的指标,以减轻标签不平衡。具体而言,训练集都是混合后的中间域数据,记$N_{all}$代表训练集中所有已标注的点数,$N_c$则为其中属于类别c的点的总数。而$N_{mix}$,$i$则代表候选子集$\mathbf{q}_i$与匹配的目标域标注点的的总点数,$N_{mix}^c$表示其中真实标签为类别c的点数。为确定是否应选择候选子集$\mathbf{q}_i$来实现更平衡的类别分布,通过将已选择的标注点总数$N_{all}$与当前选子集$\mathbf{q}_i$与匹配的目标域混合后的总点数相加,计算类别c的相对比例$\{R_{i,c}\}^C_{c=1}$,公式如\eqref{eq:class_rate}所示:
    \begin{equation}
        \label{eq:class_rate}
        R_{i,c} = \frac{N_c+N^c_{mix}}{N_{all}+N_{mix}}
    \end{equation}
    接着,为了保证在同一个量纲下比较,对$R_{i,c}$使用softmax函数进行归一化处理,得到$\hat{R}_{i,c}$,如公式\eqref{eq:class_softmax}所示:
    \begin{equation}
        \label{eq:class_softmax}
        \hat{R}_{i,c}=\frac{e^{R_{i,c}}}{\sum^C_{c=1}e^{R_{i,c}}}
    \end{equation}
    其表示选择候选子集$\mathbf{q}_i$与当前已标注的目标点云帧混合后,各类别所占点数的归一化概率。再然后,通过计算概率的熵值来可作为候选子集的选择评判指标,如公式\eqref{eq:class_entropy}所示:
    \begin{equation}
        \label{eq:class_entropy}
        H_i(\hat{R}_{i})=-\sum_{c=1}^{C} \hat{R}_{i,c}\log(\hat{R}_{i,c})
    \end{equation}
    其中,$H_i$表示选择候选子集$\mathbf{q}_i$混合后类别概率的熵值结果,并选择最大的一个候选子集$\mathbf{q}_{max}$作为最终混合,组成新的中间域数据集$\mathbf{I}=\{\mathbf{\hat{X},\hat{Y}}\}$,如公式\eqref{eq:mix_final}所示:
    \begin{equation}
        \label{eq:mix_final}
        \begin{aligned}
            \mathbf{\hat{X}}=concat\{X_{\mathbf{q}_{max}},X_{\mathbf{p}_i}\}
            \\
        \mathbf{\hat{Y}}=concat\{Y_{\mathbf{q}_{max}},Y_{\mathbf{p}_i}\}
        \end{aligned}
    \end{equation}
    其中$\mathbf{\hat{X}}$表示混合的点数据,$\mathbf{\hat{Y}}$则表示对应的标签。%将最终选择出的后选子集与目标域点进行混合,同时
    最后更新训练集总点数以及类别点数,公式如\eqref{eq:update_numeber}所示:
    \begin{equation}
        \label{eq:update_numeber}
        \begin{aligned}
            N_c=N_c+N^c_{mix}, \quad
            N_{all}=N_{all}+N_{mix}
        \end{aligned}
    \end{equation}
    
    \subsection{损失函数}
    % 对于后续的模型训练仅使用混合后的中间域数据进行微调训练,而混合后的数据都是含有标注的,因此本章使用交叉熵损失函数做为我们的优化损失函数,
    在后续的模型训练中,仅采用混合后的中间域数据进行微调优化。由于这些数据均包含标注信息,本章采用交叉熵损失函数作为优化策略,如公式\eqref{eq:cross_entropy}所示,其中$\hat{x}_{i} \in \hat{\mathbf{X}}$,$\hat{y}_{i} \in \hat{\mathbf{Y}}$:
    \begin{equation}
        \label{eq:cross_entropy}
        \mathcal{L}_{\text{CE}} = - \frac{1}{|I|} \sum_{i \in I} \sum_{c=1}^{C} \hat{y}_{i}^{c} \log h(\hat{x}_{i}^{c})
    \end{equation}

    \section{实验评估}
    \subsection{实验设置}
    \subsection{实验结果}
    先把4个表贴上
    \subsection{消融对比实验}
    把所有的能放的图和表全放上来
    \section{本章小结}
    小结
