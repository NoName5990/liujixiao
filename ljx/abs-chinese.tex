%中文摘要,自行编辑内容



\chapter{摘 \quad 要}
\xiaosi
点云是一种用于表示三维空间中的对象的数据结构。它由许多离散的点组成,可以通过激光扫描仪、三维相机或其他传感器来捕捉和记录。由于这些传感器的视野有限,因此需要将多个点云合并成一个更大的点云,或将点云与先前的点云模型对齐以进行比较或更新,这个过程就是点云配准。点云配准是计算机视觉和机器人视觉中的重要问题,它们可以用于许多应用,例如三维建模、机器人导航、虚拟现实和医学影像分析。早期的点云配准主要集中于计算机合成数据集,然而随着社会的发展,越来越多的研究开始关注真实场景下的点云配准。但真实场景中普遍存在图案重复、几何形状较弱的困难区域,这些困难区域往往会由于特征相似导致点匹配的错误,影响变换矩阵的估计结果。

为了解决该问题,本文提出了一种基于显著锚点几何嵌入的点云配准方法。该方法嵌入显著锚点与超点之间的几何结构,以增强点特征的差异性和区分度。即使在源点云和目标点云中存在大量图案重复和弱几何区域,也能分辨出相似非重叠区域找到正确的点匹配。具体来说,首先通过锚点定位模块,在源点云和目标点云中定位识别能力最强、几何信息最丰富的超点对应作为显著锚点对应。采用非最大抑制算法,保证选定的一组显著锚点在点云中稀疏分布并且具有一定的几何结构。针对显著锚点,提出了一种基于锚点距离和角度的选择性几何结构嵌入算法,用于超点特征增强。这种显著锚点与超点之间的几何一致性,可以提高几何挑战性区域的特征区分度。然后,迭代更新以增强特征和锚点位置,获得最有效的显著锚点和超点特征。最后,通过在超点对应区域内寻找最近的相邻点来实现精确的点对应。

另外,本文提出了一种基于多模态融合的锚点定位点云配准方法,该方法通过将点云的结构特征和图像的纹理特征融合以提高几何挑战性区域的特征差异性。首先本文利用对齐模块将点云和图像数据对齐以找到超点与像素之间的对应关系。然后,利用融合模块将超点与对应像素之间的特征进行融合。该融合模块将点云特征和图像特征分别投影至模态无关和模态相关的两个子空间中,并先后在两个子空间中融合两种模态特征以达到减小域差异影响和防止信息丢失的作用。\\

% 学位论文是研究生从事科研工作的成果的主要表现,集中表明了作者在研究工作中获得的新发明、新理论或新见解,是研究生申请硕士或博士学位的重要依据,也是科研领域中的重要文献资料和社会的宝贵财富。
% 为进一步规范我校研究生学位论文撰写格式,提高研究生学位论文质量,参照国家标准《学位论文编写规则》(GB/T 7713.1-2006),结合我校实际,制定本模板。
\noindent\songti\textbf{关键词:} 点云配准,几何嵌入,弱几何区域,重复图案,多模态融合

\clearpage