%中文摘要,自行编辑内容

\chapter{摘 \quad 要}
\xiaosi
点云语义分割是三维场景理解的核心任务,其目标在于将场景中每个点准确地映射到预设的语义类别上,广泛应用于无人驾驶、智能机器人以及虚拟和增强现实等领域。随着深度学习技术的迅速发展,大规模点云数据的高效处理成为可能,并催生了大量优秀的开源模型。然而,尽管点云数据采集变得更加便捷,但全监督训练要求逐点标注的数据依然难以获取,因为大规模点云逐点标注是一项耗时耗力的工作。此外,由于不同采集设备和场景的差异,将源数据集上训练的模型直接应用于新目标数据集时常会出现显著的域间隙,从而导致性能下降。域自适应作为解决域间隙问题和减少对标注依赖的重要策略,目前主要采用无监督或半监督模式,虽然节省了标注成本,但往往会牺牲部分模型性能。相比之下,主动域自适应在人工成本与分割精度之间提供了更具性价比的平衡方式。然而,目前针对三维点云语义分割主动域适应尚未提出专门的主动查询策略,也缺乏将主动选择的目标域点与已标注的源域数据有效结合的方案。针对上述问题,本文对主动学习和域自适应在视觉语义分割中的研究进展和方法进行了归类总结,阐述了现有方法的优缺点。此外,进一步提出了有效的针对三维点云语义分割的域自适应的主动学习方法以及充分发挥两域潜能的主动混合方法,方法如下所示:

% 点云语义分割是三维场景理解中的一个关键任务,其旨在将点云场景中的每一个点都映射到预设的语义类别上。在无人驾驶、智能机器人、虚拟现实、增强现实等领域,点云语义分割都有着巨大的应用价值。得益于深度学习的蓬勃发展,大规模的点云数据处理变得简单而高效,涌现出一大批优秀开源模型。随着科技的进步,点云数据的获取也变得更加容易,但是这些直接获得的数据却无法直接应用到这些优秀的模型上去,因为这些模型都是在全监督模式下训练得到的,这要求点云数据必须是逐点标注的。而大规模点云逐点标注是一项耗时耗力的工作。并且由于传感器等一些设备参数的不同,直接将源数据集上训练的模型应用于新的目标数据集时,往往会出现显著的域间隙,导致模型性能下降。域自适应是解决域间隙和数据标注的方法之一。在三维点云语义分割中,域自适应主要以无监督和半监督模式为主,尽管能减少对标注的依赖,却也不可避免地牺牲部分模型性能。相比之下,主动域自适应在人工成本与分割精度之间提供了更具性价比的平衡方式。然而,目前针对三维点云语义分割尚未提出专门的主动查询策略,也缺乏将主动选择的目标域点与已标注的源域数据有效结合的方案。针对上述问题,本文对主动学习和域自适应在视觉语义分割中的研究进展和方法进行了归类总结,阐述了现有方法的优缺点。此外,进一步提出了有效的针对三维点云语义分割的域自适应的主动学习方法以及充分发挥域适应两个数据集潜能的结合方式,方法如下所示:

% 得益于深度学习的蓬勃发展,大规模的点云数据处理变得简单而高效,加之一些高质量的开源标注点云数据集的发布,为一些卓越模型的出现铺垫好了道路。然而,这些模型都是基于开源的标记的数据集进行训练的即全监督训练,对于点云语义分割任务来说,数据的标注是一项极其耗时耗力的工作,而面对复杂多变的实际场景,显然是不现实的。而如果直接将训练好的模型直接应用于新的数据集上,由于数据获取的设备和场景的不同会有域间隙的存在,这会导致在源数据集上训练好的模型性能将会下降。因此如何解决域间隙这一绕不开的问题自然而然成为一些研究者们的关注点。域自适应便是解决域间隙的一个主流策略。
% 对于三维点云语义分割任务,域自适应更多的是无监督、半监督模式下的,而这两种方式都与全监督有一点的差距,虽然节省了标签但同时也牺牲了性能,主动域自适应可以综合平衡人工花费和性能,是一种高性价比的方式,然而在三维点云语义分割领域并没有专门的主动查询策略被提出。并且没有将主动选择的目标点和有标记的源域数据有效的结合使用起来。针对上述问题,本文对主动学习和域自适应在视觉语义分割中的研究进展和方法进行了归类总结,阐述了现有方法的优缺点。此外,进一步提出了有效的针对三维点云语义分割的域自适应的主动学习方法以及充分发挥域适应两个数据集潜能的结合方式,方法如下所示:

1. 基于点云语义分割域适应的主动学习方法,其主要贡献如下:1)提出一种原型指导的域差异感知查询策略,通过计算目标域点云特征与源域类别原型中心的余弦距离,构建"特征距离-预测熵"联合评估指标,实现跨域场景下高迁移价值点的选择。2)首次将主动学习与混合方法进行结合并用于点云语义分割域适应任务,通过混合策略将主动学习选择的目标点与源域点进行混合,构建出强健的中间域数据,进一步缩减域间隙。
% 2)设计平衡式跨域混合增强机制,通过尺度感知的空间-语义对齐技术,将新标注目标域样本与源域数据进行语义级混合,有效解决主动学习过程中的域偏移累积问题。

2. 基于点云语义分割域适应的主动混合方法,其主要贡献如下:1)提出一种源-目标数量平衡算法,混合标注数量相同的源和目标域点,使得模型可以学到更加平衡的两域知识,解决模型学习过程中的域偏移累积问题。2)在源-目标数量平衡模块的基础上提出类别平衡主动混合算法,解决混合中间域类别不平衡问题,进一步提升模型的性能。

本文提出的方法在合成到真实,真实到真实的跨域任务上做了大量实验和验证。通过与现有的同类优秀域自适应方法进行对比,实验结果和可视化结果充分证明了方法的优越性,在极少数标签的情况下达到了超越全监督的效果。\\
% 本文提出的方法在多个用于点云语义分割域适应任务的主流室外数据集上进行了大量实验和验证,包括SemanticKITTI、SemanticPOSS、nuScenes、SynLiDAR,其实验场景涵盖了合成到真实,真实到真实的跨域任务。通过与现有的同类优秀域自适应方法进行对比,实验结果和可视化结果充分证明了方法的优越性,在极少数标签的情况下达到了超越全监督的效果。\\

% 学位论文是研究生从事科研工作的成果的主要表现,集中表明了作者在研究工作中获得的新发明、新理论或新见解,是研究生申请硕士或博士学位的重要依据,也是科研领域中的重要文献资料和社会的宝贵财富。
% 为进一步规范我校研究生学位论文撰写格式,提高研究生学位论文质量,参照国家标准《学位论文编写规则》(GB/T 7713.1-2006),结合我校实际,制定本模板。
\noindent\songti\textbf{关键词:}点云处理,三维视觉,语义分割,域适应,主动学习,混合方法

\clearpage