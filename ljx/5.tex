\chapter{总结与展望}
\thispagestyle{others}
\pagestyle{others}
\xiaosi

\section{主要结论}
% 背景->问题->引出本文
% 其实没有具体的要求,就是写一下点云语义分割的应用,然后再写一下问题,主要是把问题描述的具体一点。
% 然后再写针对上述的问题,本章做了什么研究
% 点云语义分割是三维场景感知的关键一环,也是实现自动驾驶、机器人、增强现实等领域高级智能的一项重要技术。然而,由于目前优秀的深度学习模型都是基于逐点标注的数据训练而来的,但是对点云进行逐点标注却非常的困难,这阻碍了这些优秀算法的落地。因此,如何在无标注或者少量标注下得到一个性能优秀的模型变得非常重要。域适应方法通过将一个全标注源域训练得到的模型迁移到未标注的目标域上去,从而实现对新目标数据的学习同时降低标注。主动学习则是通过对选择一些最具价值的点进行标注而后对模型进行训练。而主动域适应则时结合了主动学习和域适应方法,得到更有效的方案。尽管现在的方法已经在二维图像语义分割领域取得了一定的成果,但是对于三维点云的研究却缺乏探索:
% 首先,目前的一些主动域适应的方法都是在二维图像领域的,对于三维点云语义分割域适应却缺乏研究,而三维数据与二维数据有所区别,且数量级也不同,因此一些适用于图像的主动域适应无法直接应用到点云上。
% 其次,传统的主动学习方法多是用于单域数据的,并未考虑两域间的域偏移问题,因此如果直接应用到域适应场景下,可能会导致选择次优点,从而无法发挥其最佳作用。
% 第三,未充分利用标注后的目标数据和全标注的源域数据,发挥其最大价值。由于域偏差的存在,且主动学习标注的目标点远少于源域点, 因此在适应的过程中可能导致域偏移的积累,导致模型偏向源数据,从而导致模型学习到次优的效果。
% 针对上述分析的问题,通过大量阅读相关文献并进行了实验验证,本文的主要研究内容归结如下:
点云语义分割是实现三维场景感知的关键技术,也是自动驾驶、机器人和增强现实等领域实现高级智能的重要基础。然而,目前多数优秀深度学习模型均依赖于逐点标注数据进行训练,而点云逐点标注工作极为繁琐且耗时,这成为制约这些模型实际应用的主要障碍。因此,在无标注或仅有少量标注数据的条件下获得高性能模型,显得尤为重要。
% 域适应方法通过将在全标注源域上训练得到的模型迁移到未标注目标域,从而实现对新数据的学习并降低标注成本。
域适应方法旨在将基于全标注源域训练得到的模型迁移至未标注的目标域,以实现对新数据的学习并降低标注成本。
主动学习则通过选择最具价值的样本进行标注以优化模型训练效果,而主动域适应则融合了二者的优势,提供了一种更为有效的解决方案。尽管该方法在二维图像语义分割领域已取得一定成果,但针对三维点云的研究仍显不足,主要存在以下问题:
第一,目前多数主动域适应方法集中于二维图像领域,因三维点云数据在结构和数量级上均存在显著差异,直接移植图像方法往往难以奏效。
第二,传统主动学习方法通常针对单一数据域设计,未能充分考虑源域与目标域之间的域偏移问题,因此在域适应场景下直接应用可能导致样本选择次优,影响最终效果。
第三,现有方法未能充分整合标注后的目标数据与全标注的源域数据,因主动学习选取的目标点数量远低于源域数据量,容易在迁移过程中积累域偏移,使模型倾向于源域,从而难以获得最优性能。

基于上述问题,通过大量文献阅读和实验验证,本文的主要研究内容可归纳如下:

% 1)通过大量阅读相关文献,总结并分析了目前相关领域的研究现状和进展,对所研究领域有一个系统的了解。包括点云语义分割方法,主动学习方法域自适应方法以及分割中的Mixing方法。通过上述大量文献的阅读,并根据个人理解和观点对现有方法进行归类总结,分析总结了现有方法的优点和不足,同时,对这些问题想出诸多有意义的研究思路和解决方案。
1)通过大量阅读相关文献,对点云语义分割方法、主动学习、域自适应以及分割中的混合方法等相关领域的研究现状和进展进行了系统总结和分析。基于个人的理解与观点,对现有方法进行了归类,总结了各自的优势与不足,并提出了一些具有实际意义的研究思路和解决方案。

2)对点云语义分割、主动学习和域自适应相关的基础背景知识做了详细总结。包括点云的常见表示方式,点云语义分割任务的概念、常见模型以及评估指标,主动学习的基本概念和流程以及域自适应任务的概念、目标以及常见的域对齐方式等。并对跨域点云语义分割相关任务中常用公开的数据集进行了介绍整理。

3)提出了一种基于点云语义分割域适应的主动学习方法。该方法通过构建出的源域原型,指导目标数据的选择,同时结合不确定性,筛选出兼具高不确定性和域差异性的目标点,用以高效提升模型的性能。随后将这些选择后的目标点进行标注,同时结合Mixing方法,构建兼具双域信息的中间域数据,帮助模型学习到更加鲁棒的域不变特征,进一步缩小域间隙。文中首先详细的介绍了该方法的研究动机和贡献,接着详细的阐述了该方法中的三个模块,包括源域原型构建模块、原型指导的数据选择模块、动态中间域构建模块。最后在合成到真实以及真实到真实的4个跨域数据集上,对提出的方法进行大量实验和可视化分析证明了该方法的有效性。

4)提出了一个基于点云语义分割域适应的主动混合方法。该方法在第三章方法框架基础上,深入探索了域适应场景下主动学习与Mixing方法的结合,并将动态中间域构建模块更改为了两个模块:源-目标数量平衡模块和类别平衡动态混合模块。该方法改善了模型迁移过程中因源域数据过多而导致的模型学到的知识偏向源域的问题,同时减缓了主动学习类别不平衡问题。
最后,在合成到真实以及真实到真实的四个跨域数据集上,对该方法进行了大量实验和可视化分析,从而充分验证了其有效性。
% 最后同样在合成到真实以及真实到真实的4个跨域数据集上,对该方法进行了大量实验和可视化分析以证明了其有效性。

\section{研究展望}
本文的研究中,对于所提出的两个方法分别解决了不同的问题。基于点云语义分割域适应的主动学习方法中,提出了一种适用于点云语义分割域适应的主动学习方法,
% 弥补了三维点云领域的研究空缺,
在结合Mixing后发挥最大优势,超越传统主动学习方法。基于点云语义分割域适应的主动混合方法,则是在上个方法的框架下,对点云语义分割域适应场景下主动学习与Mixing的结合进行了深入研究和探索,通过构建数量及类别平衡的中间域数据缓解了迁移过程中的域偏移累积问题和类别不平稳问题。虽然上述研究在与同领域方法的比较中取得了先进的表现,但是受限于时间、设施等多种因素,仍然存在很大的局限和提升空间。同时,基于对目前前沿技术的理解以及所做的大量实验验证分析,归结了一些点云语义分割域自适应领域中仍存在的需要进一步解决或者缓解的问题。为此,本文对未来研究做出了如下展望:

% 1)目前本研究的实验的跨域场景虽然涵盖了合成到真实以及真实到真实两个场景,但是这些场景都是基于室外数据进行的,还未对室内场景下跨域任务进行尝试,而室外和室内数据之间密度和场景大小均有差异,因此本文方法不能保证在室内跨域场景下也同样表现良好,因此提升模型的泛化能力,使其在室内外跨域数据集上能表现良好是未来需要探索的一个重要方向。
1)模型泛化能力提升。目前,本研究的跨域实验虽然涵盖了合成到真实和真实到真实两个场景,但均基于室外数据集进行的测试,尚未在室内场景下验证。由于室外与室内数据在密度和场景规模上存在显著差异,本文方法在室内跨域任务中的适用性尚无保障。因此,未来研究需着力提升模型的泛化能力,以确保其在室内外跨域数据集上均能获得良好表现,这将是一个重要的探索方向。

% 2)进一步利用未标注数据方面,虽然本文通过主动学习方法已经筛选出数量较少但价值极高的目标点,但剩余约98\%以上的目标点仍未得到利用。未来工作应探索将主动学习与伪标签技术相结合的方法,以充分挖掘大量未标注数据的潜力,提升数据利用率,并进一步增强模型的泛化性能和分割效果。这将为降低标注成本和实现高效跨域点云语义分割提供新的研究思路和技术支持。
2)进一步挖掘未标注数据的潜在价值。尽管本研究通过主动学习策略筛选出极少的高价值目标域样本进行标注,但占总量99\%的未标注点云数据仍未被有效利用。为解决这一资源浪费问题,未来研究可探索主动学习与伪标签技术的协同优化框架:首先通过构建基于不确定性和域间差异的双重评估机制,对未标注点云进行可信度分层,筛选出置信度高于阈值的点生成伪标签;同时设计对抗噪声干扰的自适应校正模块,利用点云局部几何一致性特征对伪标签进行拓扑纠偏。此外,可建立动态迭代优化机制,在每轮主动学习标注新增样本后,同步更新伪标签生成模型,形成标注数据增强与伪标签质量提升的正向循环。但也需要特别关注伪标签噪声传播与域偏移累积的耦合效应。同时,通过在特征空间构建源域标注数据、目标域标注数据、目标域伪标签数据的三体平衡约束,从而在扩大训练数据规模的同时维持模型的域不变特征表达能力。

% 3)语义选择的主动混合,虽然本文提出了一个类别平衡的主动混合策略,但是基于域适应下的混合探索还远远不够。混合可以进行有选择的混合,目前只是对源域的数据进行了约束,但是目标域上的数据仍然可以通过选择混合到其他目标帧上去,可以通过语义的平衡分别从源域和目标域上选择数据而后混合。
3)语义选择的主动混合。尽管本文提出了一种基于类别平衡的主动混合方法,但在域适应条件下,对混合方法的探索仍远远不够。目前的策略仅对源域数据进行了约束,而目标域数据仍有较大改进空间。未来可以在混合过程中采用更具选择性的策略,分别从源域和目标域中依据语义平衡挑选数据,然后进行有针对性的混合,从而更充分地利用目标域信息,进一步提升模型在跨域任务中的性能。

% 4)大模型融合,当今最火的当属大模型,其具有优秀的文本理解和逻辑推理能力,目前很多领域和方法都开始引入并结合大模型,从而进一步提升结果表现,因此,而对于域适应任务中,对于大模型的结合仍然缺乏,如何根据域自适应特点以及大模型的优势来结合大模型,也是未来一个值得探索的方向。
4)大模型融合。大模型融合是当前的研究热点,因其卓越的文本理解和逻辑推理能力,已在多个领域取得显著成效。在最近的研究中,虽然有越来越多的方法开始引入大模型以提升整体表现。然而,在语义分割域适应任务中,大模型的应用仍然缺乏深入探索。如何根据域自适应的特点,充分利用大模型的优势,构建更高效的跨域融合策略,并为提升模型性能提供新的方向,这也是未来值得重点关注和研究的课题。
