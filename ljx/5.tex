\chapter{总结与展望}
\thispagestyle{others}
\pagestyle{others}
\xiaosi

\section{主要结论}
随着科技的发展,计算机三维视觉与生产生活联系的越来越紧密,对点云的自动化处理成为了一个急切的需求。在自动驾驶的城市建图,在机器人领域的姿态估计以及虚拟现实中的三维建模,点云配准的应用广泛分布于各个生产生活环节。如何快速高效的实现各种复杂真实场景的点云配准是一项具有挑战和意义的研究。针对点云配准任务,本文提出了一种基于显著锚点的点云配准框架,它通过第三章的几何嵌入增加弱几何区域的特征差异性,通过引入第四章的多模态融合模块增加锚点的显著性。其中第四章可以看成是

针对第三章方法锚点选择的局限性做的进一步改进。两个方法都在真实场景的点云配准数据集上进行了实验和评估,实验结果表明本文提出的方法能够有效地在低重叠率得场景中,增强特征间的差异性提高相似区域重复模式的匹配成功率,并实现了当前点云配准方法的先进水平。两个方法得贡献如下:

(1)在基于显著锚点的点云配准方法中,通过提取显著锚点利用锚点与超点、超点与超点间的距离和角度等几何结构信息进行特征嵌入,增加了相似不重叠区域的差异性,能够有效提高超点匹配的内点率。该方法使用KPConv网络来提取点云局部区域的超点特征。利用一个锚点定位模块选取出若干保持一定几何结构的高置信度的锚点,并通过注意力机制对点云中的超点进行结构嵌入并寻找超点间的对应关系。最后,在经过将超点对应扩充为点对应之后,利用一个局部到全局的姿态估计得到最终的变换矩阵。

(2)在基于多模态特征融合的锚点定位点云配准方法中,通过融合点云和图像两种模态的信息,提高锚点选择的可靠性。该方法首先利用对其模块,将不同模态的两种数据进行对齐,寻找到点云到像素的对应关系。在多模态融合模块,将两种模态的特征分解为模态相关和模态无关的特征,并在模态无关子空间中缩小特征间的域间隙减少噪声干扰。并最终与模态相关的特征融合以减少信息的丢失形成最终的特征,实现锚点定位。

\section{研究展望}
本文提出了基于显著锚点几何嵌入的点云配准和基于多模态融合的锚点定位点云配准方法,虽然两个方法都在低重叠率的真实场景数据集上取得了不错的效果,但是仍然存在一些可以改进的地方。
本文的第三章提出了一种基于显著锚点几何嵌入的点云配准方法,其核心思想是通过多个保持一定几何机构的锚点缓解相似不重叠区域特征过度平滑问题。虽然该方法取得了一定的效果,但是仍然存在一些尚需改进的地方:(1)首先该方法虽然设计了一个锚点定位模块并利用迭代优化更新显著锚点,但是这种设计产生的锚点在某些场景中依然会失败,并导致最终的结果相较于一般方法较差。为此,需要设计一个更加鲁棒的锚点定位模块使得整体网络更加稳定,可以考虑设计一个损失函数来有效监督锚点对应。(2)其次在使用由粗到细的点云配准框架之后,整个网络的模型较大导致训练时间较长,如何有效轻量化模型是一个急需解决的问题。后续可以通过提高下采样倍率减少超点个数,进而减少几何嵌入过程的时间开销。

本文的第四章提出以一种基于多模态融合的锚点定位点云配准方法,其核心思想是通过将图像信息和点云信息进行特征融合提高锚点对应的准确性。该方案针对第三章框架做出一点改进并取得了一定效果,但也存在一些问题:(1)点云数据集中每个点云实际是由50张RGB-D图像合成,而文章中仅采取某一视角下的一张图片进行融合导致某些点在该视角下被遮挡找不到准确的图像信息,但是如果使用全部图像又会导致时间花销较大,故需要有效实验对图像数量对模态融合的影响进行分析。(2)在多模态融合模块中,将两种模态的特征投影到模态相关和模态无关的两个子空间并分别学习模态相关与无关的特征表示,后续工作可以重新设计更加适应点云和图像融合的损失函数对其进行监督以达到预设效果。同时在最终的融合过程中使用多头注意力机制完成,后续工作可以考虑使用其他的多模态融合方法。

如何增加点特征间的差异性缓解特征的过渡平滑是点云配准任务的关键,提出的两个方法都是借助锚点嵌入几何信息来提取最终特征。但是由于点云点的个数较多,在做特征提取时导致时间和空间花销较大,影响了整个模型的训练效率。在后续工作中,可以考虑使用与训练好的特征提取网络来提前提取好特征,训练时则读取相应的特征,以减少训练过程中的时间花销是整个网路的参数量大大减少。同时随着点云和图像多模态融合的研究越发深入,对于这两种模态如何更好地融合有了更多的考量。相比于直接利用图像特征修饰点云,使用特征间的融合更加有效;相比于特征间的隐式融合,通过对齐两种模态的显式特征融合更加优越,后续工作可以对点云和图像特征的融合方式开展广泛的实验研究。同时为了能够将现有网络模型部署到例如火星探测等相关实验平台,设计轻量化的网络结构也是一种新的研究方向。