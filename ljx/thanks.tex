% 致谢
%\specialsectioning
\chapter{致 \quad 谢}
\thispagestyle{others}
\pagestyle{others}
\xiaosi
% 在硕士三年的求学时光里,我经历了疫情带来的封闭与压抑,也曾因一次次实验失败而自我怀疑,但这一切都因那些在关键时刻给予我帮助与支持的人而变得意义非凡。首先,我要深深感谢我的导师高xx教授和徐xx副教授,是他们引领我走入科研的殿堂,一步步教会我如何阅读文献、发现问题并寻求解决之道。同时感谢重庆邮电大学图像认知重点实验室,给我提供了优质的科研平台和设备,保证我的实验能够正常进行。其次,感谢我的家人,是他们无私的关怀与鼓励,在每次实验挫折之后给我力量,让我能够重新振作、继续前行。最后,感谢我所在团队中的每一位成员和实验室的同学,你们的帮助与陪伴,让我的科研道路不再孤单。科研之路,始于重邮,终于重邮。也感谢各位毕业答辩老师,在我步入社会前给予学生时代的最后一课。言简意赅,字少情真,愿未来的人生画卷中,我们都能勾勒出自己最绚丽的风景。
% 在硕士求学的三年里,我经历了疫情带来的封锁与压抑,也在一次次实验失败后产生过自我怀疑。正是在那些关键时刻,许多人的帮助和支持让我找回了信心,赋予了这段时光非凡的意义。
2022年9月21日,我从山东梁山的一个小村庄踏上了去往重庆的火车,同时也踏上了我的科研路。那时的我满怀期待与热血,渴望进入科学的殿堂,能近距离接触科研,做出一番成就。三年里,我经历了疫情带来的封锁与压抑,也在一次次实验失败后产生过自我怀疑。我曾多次辗转反侧于床上,想要抓住科研灵感,也曾数次陪伴晚上2点的路灯与汽车修改实验和代码。然而每一次的坎坷都是成功路上的垒石,当实验开始展现效果、当小论文写成之时,心中涌动的喜悦难以言表。不知不觉,三年已过,虽没有做出大名堂,但也尽其心力,无愧自我,将自己三年的心血凝聚在一篇小论文和此文中。成文之际,希望在此感谢那些在我科研路上曾为我提供帮助的人,是他们给予我再来的勇气。
% 然而,每一次的失败都有许多人的帮助和支持让我找回了信心,赋予了硕士生涯不可磨灭的记忆,在此我想要哪些。

首先,我要衷心感谢我的导师高新波教授和徐宗懿副教授。感谢他们引领我走入科学研究的殿堂,教会我如何阅读前沿文献、洞察研究问题,并耐心指导我探索解决思路。您们严谨的治学态度和执着的科研精神,将成为我未来道路上的指路明灯。其次,我要特别感谢我的家人。是他们无私的关怀与鼓励,让我在挫折面前重新振作,是他们默默的付出和耐心倾听,使我在疲惫时依然充满力量,继续朝着目标前行。最后,感谢重庆邮电大学图像认知重点实验室为我提供了优越的平台与先进的设备,使我的实验得以顺利开展;感谢实验室所有师兄师弟和同门伙伴,在我遇到难题时慷慨分享经验,在我情绪低落时温暖陪伴,让科研之路不再孤单。
% 最后,感谢各位毕业答辩老师,在即将步入社会之际,为我圆满学生时代的最后一课。

科研之路,始于重邮,亦将归于重邮。愿我们在未来的人生画卷中,都能绘就最绚丽的风景。
