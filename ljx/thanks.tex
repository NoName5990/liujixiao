\chapter{致 \quad 谢}
\thispagestyle{others}
\pagestyle{others}
\xiaosi
2022年9月21日,我从山东梁山的一个小村庄踏上了去往重庆的列车,也由此开启了我的科研之旅。彼时,我怀揣热忱,渴望走进科学殿堂,近距离参与科研并取得成绩。三年来,我经历了疫情带来的封锁与压抑,也在一次次实验失利后产生自我怀疑。多少个夜晚,我在床上辗转反侧,努力捕捉科研灵感;多少次深夜两点,台灯下与电脑代码相伴。然而,正是这些坎坷,成为通往成功的垫脚石,当实验初见成效,当小论文完成之时,心中涌动的喜悦难以言表。转眼三年虽未做出惊世之举,但我倾注心血,将所有努力凝练于小论文与此文中。值此成文之际,我衷心感谢在科研路上给予我帮助与支持的每一位良师益友,是你们赋予我执着奋斗的勇气。

首先,我要衷心感谢我的导师高新波教授和徐宗懿副教授。感谢他们引领我走入科学研究的殿堂,教会我如何阅读前沿文献、洞察研究问题,并耐心指导我探索解决思路。您们严谨的治学态度和执着的科研精神,将成为我未来道路上的指路明灯。其次,我要特别感谢我的家人。是他们无私的关怀与鼓励,让我在挫折面前重新振作,是他们默默的付出和耐心倾听,使我在疲惫时依然充满力量,继续朝着目标前行。最后,感谢重庆邮电大学图像认知重点实验室为我提供了优越的平台与先进的设备,使我的实验得以顺利开展;感谢实验室所有师兄师弟和同门伙伴,在我遇到难题时慷慨分享经验,在我情绪低落时温暖陪伴,让科研之路不再孤单。

科研之路,始于重邮,亦将归于重邮。愿我们在未来的人生画卷中,都能绘就最绚丽的风景。
