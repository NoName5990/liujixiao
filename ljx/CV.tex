% \specialsectioning


% \chapter{作者简介}
% \thispagestyle{others}
% \pagestyle{others}
% \xiaosi

% \section{1. \ 攻读学位期间的研究成果}
% \subsection{(一) \ 发表的学术论文和著作}
% \noindent [1]
% \begin{minipage}[t]{0.96\linewidth}
% 第2作者(导师为第1作者). IEEE Transactions on Circuits and Systems for Video Technology.(在投)
% \end{minipage}
% \vspace{0cm}


% \subsection{(二) \ 申请(授权)专利}
% \noindent [1]
% \begin{minipage}[t]{0.96\linewidth}
% 第2作者(导师为第1作者). 2023.01.29.
% \end{minipage}

% \noindent [2]
% \begin{minipage}[t]{0.96\linewidth}
% 第2作者(导师为第1作者). 2023.02.24.
% \end{minipage}

% \subsection{(三) \ 参与的科研项目及获奖}

\specialsectioning


\chapter{作者简介}
\thispagestyle{others}
\pagestyle{others}
\xiaosi
\vspace{-0.1cm}
\section{1. \ 基本情况}
\vspace{-0.1cm}
% 张某某,男,重庆人,1993年8月出生,重庆邮电大学XX学院XX专业2018级博士研究生。
刘继逍,男,山东济宁人,1999年1月出生,重庆邮电大学计算机科学与技术学院计算机科学与技术专业2022级硕士研究生。

\section{2. \ 教育和工作经历}
\vspace{-0.1cm}
\begin{description}[leftmargin=3.5cm, style=sameline]
\item[2017.09$\sim$2021.07] 内蒙古科技大学信息工程学院,本科,专业:软件工程
% \vspace{-0.5cm}
\item[2022.09$\sim$2025.06] 重庆邮电大学计算机科学与技术学院,硕士研究生,专业:计算机科学与技术
\end{description}

\section{3. \ 攻读学位期间的研究成果}
\vspace{-0.1cm}
\subsection{3.1 \ 发表的学术论文和著作}
\noindent [1]
\begin{minipage}[t]{0.96\linewidth}
% 第2作者(导师为第1作者). ACM International Conference on Multimedia.(在投)
Xu Z, \textbf{Liu J}, Zhao S, et al. Domain Discrepancy Aware Active Learning for Cross-domain LiDAR Point Cloud Segmentation[C]. ACM International Conference on Multimedia.(在投)
\end{minipage}

\subsection{3.2 \ 申请(授权)专利}
\vspace{-0.1cm}
\noindent [1]
\begin{minipage}[t]{0.96\linewidth}
% 第2作者(导师为第1作者). 一种基于原型指导的跨域主动学习的点云分割方法. 2025.02.18.
徐宗懿,\textbf{刘继逍},郎忠堋,高新波.一种基于原型指导的跨域主动学习的点云分割方法$:$ \ 202510175684.7[P]. 2025.2.18.
\end{minipage}
\noindent [2]
\begin{minipage}[t]{0.96\linewidth}
徐宗懿,王飞,高震,高鑫雨,\textbf{刘继逍},高新波.一种基于单张图像的高质量三维服装生成方法$:$ \ CN202410449279.5[P]. 2024.04.15.
\end{minipage}
\noindent [3]
\begin{minipage}[t]{0.96\linewidth}
徐宗懿,王飞,高震,高鑫雨,\textbf{刘继逍},高新波.一种面向三维服装的可控参数模型构建方法$:$ \ CN202410446771.7[P]. 2024.04.15.
\end{minipage}
\noindent [4]
\begin{minipage}[t]{0.96\linewidth}
徐宗懿,王飞,高震,高鑫雨,\textbf{刘继逍},高新波.一种面向三维服装模型的非刚性配准方法$:$ \ CN202410435939.4[P]. 2024.04.11. 
\end{minipage}
\noindent [5]
\begin{minipage}[t]{0.96\linewidth}
% 第3作者(导师为第1作者). 一种基于选点主动学习的半监督点云语义分割方法. 2023.08.04.
徐宗懿,袁波,\textbf{刘继逍},高新波.一种基于选点主动学习的半监督点云语义分割方法$:$ \ CN202310497117.4[P]. 2023.05.05.
\end{minipage}
\noindent [6]
\begin{minipage}[t]{0.96\linewidth}
% 第3作者(导师为第1作者). 点云样本选择方法、装置及计算机设备. 2023.08.04.
徐宗懿,袁波,\textbf{刘继逍},高新波.点云样本选择方法、装置及计算机设备$:$ \ CN202310497177.6[P]. 2023.05.05.
\end{minipage}

\subsection{3.3 \ 参与的科研项目及获奖}
\vspace{-0.1cm}
\noindent [1]
\begin{minipage}[t]{0.96\linewidth}
    国家自然科学基金委员会,青年科学基金项目: 面向下一代火星车三维视觉系统的点云配准研究(No.62206033),2023.01.01-2025.12.31, 参与.
\end{minipage}
\noindent [2]
\begin{minipage}[t]{0.96\linewidth}
    国家自然科学基金委员会,国家自然科学基金创新研究群体项目: 多粒度认知计算理论及应用(No.62221005),2023.01-2027.12, 参与.
\end{minipage}