% \specialsectioning


% \chapter{作者简介}
% \thispagestyle{others}
% \pagestyle{others}
% \xiaosi

% \section{1. \ 攻读学位期间的研究成果}
% \subsection{(一) \ 发表的学术论文和著作}
% \noindent [1]
% \begin{minipage}[t]{0.96\linewidth}
% 第2作者(导师为第1作者). IEEE Transactions on Circuits and Systems for Video Technology.(在投)
% \end{minipage}
% \vspace{0cm}


% \subsection{(二) \ 申请(授权)专利}
% \noindent [1]
% \begin{minipage}[t]{0.96\linewidth}
% 第2作者(导师为第1作者). 2023.01.29.
% \end{minipage}

% \noindent [2]
% \begin{minipage}[t]{0.96\linewidth}
% 第2作者(导师为第1作者). 2023.02.24.
% \end{minipage}

% \subsection{(三) \ 参与的科研项目及获奖}

\specialsectioning


\chapter{作者简介}
\thispagestyle{others}
\pagestyle{others}
\xiaosi

\section{1. \ 基本情况}
% 张某某,男,重庆人,1993年8月出生,重庆邮电大学XX学院XX专业2018级博士研究生。
刘继逍,男,山东济宁人,1999年1月出生,重庆邮电大学计算机科学与技术学院计算机科学与技术专业2022级硕士研究生。

\section{2. \ 教育和工作经历}
\begin{description}[leftmargin=3.5cm, style=sameline]
\item[2017.09$\sim$2021.07] 内蒙古科技大学信息工程学院,本科,专业:软件工程
\item[2022.09$\sim$2025.06] 重庆邮电大学计算机科学与技术学院,硕士研究生,专业:计算机科学与技术
\end{description}
% \begin{tabular}{@{}p{4.5cm}p{\dimexpr\textwidth-4.5cm\relax}@{}}
%     2017.09~2021.07 & 内蒙古科技大学信息工程学院,本科,专业:软件工程 \\
%     2022.09~2025.06 & 重庆邮电大学计算机科学与技术学院,硕士研究生,专业:计算机科学与技术 \\
%     \end{tabular}
% 2017.09$\sim$2021.07 内蒙古科技大学信息工程学院,本科,专业:软件工程

% 2022.09$\sim$2025.06 重庆邮电大学计算机科学与技术学院,硕士研究生,专业:计算机科学与技术

\section{3. \ 攻读学位期间的研究成果}

\subsection{3.1 \ 发表的学术论文和著作}
\noindent [1]
\begin{minipage}[t]{0.96\linewidth}
第2作者(导师为第1作者). ACM International Conference on Multimedia.(在投)
\end{minipage}
\vspace{0cm}

\subsection{3.2 \ 申请(授权)专利}
\noindent [1]
\begin{minipage}[t]{0.96\linewidth}
第2作者(导师为第1作者). 一种基于原型指导的跨域主动学习的点云分割方法. 2025.02.18.
\end{minipage}
\noindent [1]
\begin{minipage}[t]{0.96\linewidth}
第3作者(导师为第1作者). 一种基于选点主动学习的半监督点云语义分割方法. 2023.08.04.
\end{minipage}
\noindent [1]
\begin{minipage}[t]{0.96\linewidth}
第3作者(导师为第1作者). 点云样本选择方法、装置及计算机设备. 2023.08.04.
\end{minipage}

% \subsection{3.3 \ 参与的科研项目及获奖}