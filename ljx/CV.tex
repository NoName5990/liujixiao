\specialsectioning


\chapter{作者简介}
\thispagestyle{others}
\pagestyle{others}
\xiaosi
\section{1. \ 基本情况}
刘继逍,男,山东济宁人,1999年1月出生,重庆邮电大学计算机科学与技术学院计算机科学与技术专业2022级硕士研究生。

\section{2. \ 教育和工作经历}
% \vspace{-0.1cm}
2017.09$\sim$2021.07 \ \ 内蒙古科技大学信息工程学院,本科,专业:软件工程

2022.09$\sim$2025.06 \ \ 重庆邮电大学计算机科学与技术学院,硕士研究生,专业:

\qquad \qquad \qquad \qquad 计算机科学与技术


\section{3. \ 攻读学位期间的研究成果}
\subsection{\quad 3.1 \ 发表的学术论文和著作 }
\noindent [1]
\begin{minipage}[t]{0.96\linewidth}
Xu Z, \textbf{Liu J}, Zhao S, et al. Domain Discrepancy Aware Active Learning for Cross-domain LiDAR Point Cloud Segmentation[C]. ACM International Conference on Multimedia.(在投)
\end{minipage}

\subsection{\quad 3.2 \ 申请(授权)专利}
\noindent [1]
\begin{minipage}[t]{0.96\linewidth}
徐宗懿,\textbf{刘继逍},郎忠堋,高新波.一种基于原型指导的跨域主动学习的点云分割方法:202510175684.7[P]. 2025.2.18.
\end{minipage}
\vspace{0.20pt}

\noindent [2]
\begin{minipage}[t]{0.96\linewidth}
徐宗懿,袁波,\textbf{刘继逍},高新波.一种基于选点主动学习的半监督点云语义分割方法:CN202310497117.4[P]. 2023.05.05.
\end{minipage}
\vspace{0.20pt}

\noindent [3]
\begin{minipage}[t]{0.96\linewidth}
徐宗懿,袁波,\textbf{刘继逍},高新波.点云样本选择方法、装置及计算机设备:CN202310497177.6[P]. 2023.05.05.
\end{minipage}

\subsection{\quad 3.3 \ 参与的科研项目及获奖}
\noindent [1]
\begin{minipage}[t]{0.96\linewidth}
国家自然科学基金委员会,青年科学基金项目: 面向下一代火星车三维视觉系统的点云配准研究(No.62206033),2023.01.
\end{minipage}

% \noindent 
% \begin{minipage}[t]{0.96\linewidth}
% [2] 徐宗懿,王飞,高震,高鑫雨,\textbf{刘继逍},高新波.一种基于单张图像的高质量三维服装生成方法:CN202410449279.5[P]. 2024.04.15.
% \end{minipage}

% \noindent 
% \begin{minipage}[t]{0.96\linewidth}
% [3] 徐宗懿,王飞,高震,高鑫雨,\textbf{刘继逍},高新波.一种面向三维服装的可控参数模型构建方法:CN202410446771.7[P]. 2024.04.15.
% \end{minipage}

% \noindent
% \begin{minipage}[t]{0.96\linewidth}
% [4] 徐宗懿,王飞,高震,高鑫雨,\textbf{刘继逍},高新波.一种面向三维服装模型的非刚性配准方法:CN202410435939.4[P]. 2024.04.11. 
% \end{minipage}