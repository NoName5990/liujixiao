\begin{table}[H]
	\renewcommand{\arraystretch}{1}
    \centering
    \setlength{\tabcolsep}{10mm}
    \bicaption[\xiaosi 第三章方法与其他域适应方法在SynLiDAR\(\to\)SemanticKITTI数据上的比较]{\wuhao 本方法与其他域适应方法在SynLiDAR\(\to\)SemanticKITTI数据上的比较}{\wuhao Comparison with other domain adaptation methods on SynLiDAR\(\to\)SemanticKITTI}
    \label{tab:4-1}
    \wuhao
    \begin{tabular}{lccc}
        \toprule[1.5pt]
        \textbf{方法} & \textbf{域适应} & \textbf{标注} & \textbf{结果} \\
        \midrule
        Source-Only   & -          & -       & 22.8 \\
        Target-Only   & -          & 100\%       & 60.1 \\
        AADA          & UDA & -       & 23.0 \\
        AdvEnt        & UDA & -       & 25.8 \\
        CRST          & UDA & -       & 26.5 \\
        ST-PCt        & UDA & -       & 28.9 \\
        PolarMix      & UDA & -       & 32.2 \\
        CoSMix        & UDA & -       & 31.0 \\
        DGT-ST        & UDA & -       & 43.1 \\
        MME           & SSDA & 0.04\%  & 24.5 \\
        APE           & SSDA & 0.04\%  & 25.1 \\
        APE-PCT       & SSDA & 0.04\%  & 27.0 \\
        CoSMix-SSDA   & SSDA & 0.04\%  & 34.3 \\
        本章方法       & SSDA   & 0.04\%   & 50.5 \\
        Annotator     & ADA   & 0.1\%     & 57.7 \\
        前章方法     & ADA   & 0.1\%     & 58.7 \\
        \textbf{本章方法}       & ADA   & 0.1\%     & \textbf{65.5} \\
        \bottomrule[1.5pt]
    \end{tabular}
\end{table}