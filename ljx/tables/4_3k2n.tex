% \begin{table}[htbp]
\begin{table}[H]
	\renewcommand{\arraystretch}{1}
    \centering
    \setlength{\tabcolsep}{12mm}
    \bicaption[\xiaosi 第四章方法与其他域适应方法在SemanticKITTI\(\to\)nuScenes数据上的比较]{\wuhao 本章方法与其他域适应方法在SemanticKITTI\(\to\)nuScenes数据上的比较}{\wuhao Comparison with other domain adaptation methods on SemanticKITTI\(\to\)nuScenes}
    \label{tab:4-3}
    \wuhao
    \begin{tabular}{cccc}
        \toprule[1.5pt]
        \textbf{方法} & \textbf{域适应} & \textbf{标注} & \textbf{mIoU(\%)} \\
        \midrule
        Source-Only   & -       & -           & 33.7 \\
        Target-Only   & -       & 100\%           & 82.7 \\
        Mix3D\upcite{nekrasov2021mix3d}         & UDA     & -   & 31.5 \\
        CoSMix\upcite{saltori2022cosmix}        & UDA     & -   & 29.8 \\
        SN\upcite{wang2020train}              & UDA   & -     & 25.8 \\
        RayCast\upcite{langer2020domain}        & UDA    & -    & 30.9 \\
        LiDOG\upcite{saltori2023walking}        & UDA      & -       & 34.9 \\
        前章方法       & ADA    & 0.1\%      & 76.5 \\
        Annotator\upcite{Annotator}     & ADA     & 0.1\%     & 75.9 \\
        \textbf{本章方法}       & ADA    & 0.1\%      & \textbf{81.0} \\
        \bottomrule[1.5pt]
    \end{tabular}
\end{table}