% \begin{table}[htbp]
\begin{table}[H]
	\renewcommand{\arraystretch}{1}
    \centering
    \setlength{\tabcolsep}{10mm}
    \bicaption[\xiaosi 第三章方法与其他域适应方法在SynLiDAR\(\to\)SemanticPOSS数据上的比较]{\wuhao 本方法与其他域适应方法在SynLiDAR\(\to\)SemanticPOSS数据上的比较}{\wuhao Comparison with other domain adaptation methods on SynLiDAR\(\to\)SemanticPOSS}
    \label{tab:4-2}
    \wuhao
    \begin{tabular}{lccc}
        \toprule[1.5pt]
        \textbf{方法} & \textbf{域适应} & \textbf{标注} & \textbf{结果} \\
        \midrule
        Source-Only   & -           & -       & 34.6 \\
        Target-Only   & -           & 100\%       & 58.0 \\
        CRST\upcite{xx}          & UDA & -       & 27.1 \\
        ST-PCT        & UDA & -       & 29.6 \\
        PolarMix      & UDA & -       & 30.4 \\
        CoSMix        & UDA & -       & 40.4 \\
        DGT-ST        & UDA & -       & 50.8 \\
        MME           & SSDA & 0.01\%  & 33.2 \\
        APE           & SSDA & 0.01\%  & 30.3 \\
        APE-PCT       & SSDA & 0.01\%  & 31.2 \\
        CoSMix-SSDA   & SSDA & 0.01\%  & 41.0 \\
        本章方法       & SSDA   & 0.01\%   & 57.5 \\
        Annotator\upcite{Annotator}     & ADA   & 0.1\%     & 52.0 \\
        前章方法       & SSDA   & 0.1\%   & 56.6 \\
        \textbf{本章方法}       & ADA   & 0.1\%     & \textbf{60.2} \\
        \bottomrule[1.5pt]
    \end{tabular}
\end{table}