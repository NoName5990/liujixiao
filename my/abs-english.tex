%英文摘要,自行编辑内容




\chapter{ABSTRACT}
\xiaosi
A point cloud is a data structure used to represent objects in three-dimensional space. It consists of many discrete points that can be captured and recorded by laser scanners, 3D cameras, or other sensors. Due to the limited field of view of these sensors, multiple point clouds need to be merged into a larger point cloud, or point clouds are aligned with previous point cloud models for comparison or updating. This process is point cloud registration. Point cloud registration is an important problem in computer vision and robot vision, and it can be used in many applications, such as 3D modeling, robot navigation, virtual reality, and medical image analysis. Early point cloud registration mainly focused on computer-synthesized datasets. However, with the development of society, more and more researchers began to focus on point cloud registration in real scenes. However, the geometrically challenging areas with repetitive patterns and low geometry commonly exist in real scenes, causing failure in point matching followed by inaccurate point cloud registration. 

In this thesis, this thesis propose a robust point cloud registration approach that embeds the geometry of salient anchors to enhance the discriminative ability of the point features even in the presence of a large number of repetitive patterns and low-geometry areas in the source and target point clouds. Specifically, an anchor location module is designed to locate corresponding superpoints with the most discriminative and the richest geometric information as salient anchors in the source and target. Non-maximum suppression is adopted to ensure the salient anchors are structure-preserved and sparsely distributed. With salient anchors, a selectively geometric structure embedding of anchorsuperpoint distances and angles is proposed for superpoint feature enhancement. This integration of geometry consistency between the salient anchors and superpoints can improve the distinction of features in those geometrically challenging areas. Afterwards, the enhanced features and anchor positions are updated in an iterative manner to acquire the most effective salient anchors and descriptive superpoint features. The updated features allow for accurate superpoint matches. Finally, accurate point correspondences are achieved by finding the nearest neighbour points within superpoints. 

In addition, this thesis proposes a point cloud registration method based on multimodal fusion for anchor location, which improves the feature diversity of geometrically challenging regions by fusing the structural features of the point cloud and the texture features of the image. First, this paper uses the alignment module to align the point cloud and image data to find the correspondence between superpoints and pixels. Then, a fusion module is used to fuse the features between the superpoints and the corresponding pixels. The fusion module projects the point cloud features and image features into two subspaces that are modality-independent and modality-dependent, and fuses the two modality features successively in the two subspaces to reduce the impact of domain differences and prevent information loss role.\\
% Dissertation /Thesis is postgraduate’s main academic performance to display her/his works of scientific research, which shows the author’s new invention, new theory or new opinion in her/his research. It is the crucial document for the graduate students to apply for degree, and it is also the important scientific research literature and the valuable wealth of society.

% In order to further standardize the format of dissertation/thesis writing and improve graduate dissertation/thesis quality, this temolate is formulated with reference to the national standard "Rules for Dissertation Writing" (GB/T 7713.1-2006) and the reality of CQUPT.
\noindent\textbf{Keywords:} Point cloud registration, Geometry embedding, Low-geometry area, Repetitive patterns, Multimodal fusion

\clearpage