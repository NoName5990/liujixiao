

\chapter{图表、公式格式和印制要求Abc}
\thispagestyle{others}
\pagestyle{others}
\xiaosi

\section{本章引言}
本章引言……

\section{引用参考文献}
参考文献引用示例:单篇引用\textsuperscript{\cite{ref1}},单篇多次引用\textsuperscript{\cite{ref1}55},多篇同处引用\textsuperscript{\cite{ref1,ref2,ref3,ref13}}


\section{图和表格式}

图、表在版面中居中放置,图编号和图题居中列在图下。编号采用阿拉伯数字分章连续编号,例如“图 \ref{fig:3.1}”,“表 \ref{tab:3.1}”以及“式 \ref{eq:3.1}”。

\subsection{图}
下面给出图片示例:

%调整图片与上方文字之间的间距
\vspace{-0.1cm}

\begin{figure}[h]
		\centering 
		\includegraphics[width=10cm]{chapters/31}
	    \bicaption[\xiaosi 不同缩放系数v的缩放效果]{\wuhao 不同缩放系数v的缩放结果}{\wuhao Scaling results with different scaling coefficients ν}
	   	 \label{fig:3.1}
\end{figure}

%调整图片与下方文字之间的间距
\vspace{-0.35cm}

图片标题与图片之间的间距使用默认设置即可,与上下文的间距由于LATEX动态排版特性,需要大家手动调整。

。

。

。

。

。

。

。

。



下图是多子图示例:
%\vspace{-1cm}



\begin{figure}[h]
	\centering
	\subfigure[]{
		\label{fig:DE_J}
		\includegraphics[width=12cm]{chapters/DE_J.pdf}}
	\subfigure[]{
		\label{fig:DE_CF}
		\includegraphics[width=12cm]{chapters/DE_CF.pdf}}   
\bicaption[\xiaosi 理论效率与$\gamma$和$\varphi$的关系。]{\wuhao 理论效率与$\gamma$和$\varphi$的关系。 (a) $\alpha=1$; (b) $\alpha=2/\sqrt{3}$}{\wuhao Theoretical DE versus $\gamma$ and $\varphi$. (a) $\alpha=1$; (b) $\alpha=2/\sqrt{3}$}

%	\caption{\wuhao 理论效率与$\gamma$和$\varphi$的关系。 (a) $\alpha=1$; (b) $\alpha=2/\sqrt{3}$}
%%	\raggedright
%	\wuhao Fig. 3-2 Theoretical DE versus $\gamma$ and $\varphi$. (a) $\alpha=1$. (b) $\alpha=2/\sqrt{3}$.Theoretical DE versus $\gamma$ and $\varphi$. (a) $\alpha=1$. (b) $\alpha=2/\sqrt{3}$.
\end{figure}

\vspace{-0.5cm}

\subsection{表}

表格格式参照写作指南。表格格式参照写作指南。表格格式参照写作指南。表格格式参照写作指南。表格格式参照写作指南。表格格式参照写作指南。表格格式参照写作指南。表格格式参照写作指南。表格格式参照写作指南。表格格式参照写作指南。表格格式参照写作指南。表格格式参照写作指南。表格格式参照写作指南。表格格式参照写作指南。表格格式参照写作指南。表格格式参照写作指南。

\vspace{0.1cm}

\begin{table}[h]
	\renewcommand{\arraystretch}{1.5}
	\centering
	\bicaption[\xiaosi 电流类型对效率的影响]{\wuhao 电流类型对效率的影响}{\wuhao Current type impact on efficiency}
	\begin{tabular}{p{3cm}p{3cm}p{3cm}p{3cm}}
		\toprule[1.5pt]
		\makecell[c]{\songti\wuhao 电流类型}&\makecell[c]{\songti\wuhao A}&\makecell[c]{\songti\wuhao B}&\makecell[c]{\songti\wuhao C}\\
		\hline
		\makecell[c]{\wuhao aaa}&\makecell[c]{\wuhao aa1}&\makecell[c]{\wuhao bb1}&\makecell[c]{\wuhao cc1}\\
		\bottomrule[1.5pt]
	\end{tabular}
   \label{tab:3.1} 	
\end{table}

表格格式参照写作指南。表格格式参照写作指南。表格格式参照写作指南。表格格式参照写作指南。表格格式参照写作指南。表格格式参照写作指南。表格格式参照写作指南。表格格式参照写作指南。表格格式参照写作指南。表格格式参照写作指南。表格格式参照写作指南。表格格式参照写作指南。表格格式参照写作指南。表格格式参照写作指南。表格格式参照写作指南。表格格式参照写作指南。

\vspace{-0.1cm}

\begin{table*}[h]
	\renewcommand{\arraystretch}{1.5}
	\bicaption[\xiaosi 高效率功放性能对比]{\wuhao 高效率功放性能对比}{\wuhao High-effiency power amplifier performance comparison}
	\label{tab_1}
	\centering
	\wuhao
	\begin{tabular}{c c c c c }
		\hline
		{\textbf{带宽}(GHz)}&{\textbf{功率}(dBm)}&{\textbf{效率}(\%)}&{\textbf{线性度}(dBc)}&{\textbf{信号带宽}(MHz)}\\
		\hline
		1.4--2.6&32--34&30--40 (DE)&-25 -- -30 (ACLR)&5\\
		\hline
		\multirow{2}{*}{2.1--2.7}&39&45 (DE) @ 2.14 GHz&--31 (ACLR)&\multirow{2}{*}{5}\\\cline{3-4}
		&(average)&40 (DE) @ 2.655 GHz&--30 (ACLR)&\\
		\hline
		3.5&38.1&59 (PAE)&30 (C/I)&5\\
		\hline
		\multirow{2}{*}{1.6--2.6}&36.0--38.5&45--60 (PAE)&30 (C/I)&5\\\cline{2-5}
		&35.3--37.5&40--55 (PAE)&--30 (ACLR)&20\\
		\hline
	\end{tabular}
\end{table*}


\section{公式格式}

\begin{equation}
\left\{ \begin{aligned}
0.794 \le \zeta  \le 1 ~~~~~~~~~~~\\
0.631 \le \gamma  = \frac{{0.631}}{{{\zeta ^2}}} \le 1~~~~~~ \\
- \frac{1}{{2\gamma }} \le \delta  \le \frac{1}{{2\gamma }}~~~~~~~~~~~ \\
{Z_{c,low,f}} = 2{R_{opt}}(\gamma  + j\delta )~~~~~\\
{Z_{c,2f}} = {Z_{c,low,2f}} =  - j\frac{{3\pi }}{4}\gamma \delta {R_{opt}}
\end{aligned} \right.
\label{eq:3.1}
\end{equation}

\begin{equation}
\begin{aligned}
v(\theta ) = V_{DD}\cdot(1 - \alpha cos(\theta  + \varphi ) + \beta cos(3\theta  + 3\varphi ))\\
\cdot(1 - \gamma \sin (\theta  + \varphi )) ~~~~~- 1 \le \gamma  \le 1\
\end{aligned}
\label{eq:vd}
\end{equation}




\noindent
公式格式测试。${\mathbf{\Theta }} = \left\{ {{\theta _k}\left( n \right),\forall k,n} \right\}$

\section{印制要求}
涉密学位论文的印刷、制作、传递、存档等,须符合国家、学校相关保密要求。学位论文一律左侧装订。

中文摘要之前的前置部分(封面、中、英文题名页、独创性声明和使用授权书),采用单面印刷。

从中文摘要开始,采用双面印刷。

中文摘要及之后的前置部分,包括中文摘要、ABSTRACT、目录、图目录(如有)、表目录(如有)、主要符号表(如有)、缩略词表(如有),在双面印刷时,若某部分页数为奇数,则该部分最后一页单面印刷。例如:若“摘要”只有1页,则其页码是“Ⅰ”,第“Ⅰ”页纸的背面为空白(无页眉或页码);“ABSTRACT”用新的一张纸印刷,页码从“Ⅱ”开始。

从第1章第1页开始,至论文最后1页,所有页面均双面印刷。例如:若第1章的最后1页为第17页,则第2章的第1页在第17页的背面印刷,页码为“18”(页眉是“重庆邮电大学博士(硕士)学位论文”)。

一次性双面打印整本学位论文技巧:除用于打印的版本外,电子版论文中一律不得出现空白页。论文打印建议使用PDF格式。为方便一次性双面打印,有时可在单面印刷的部分(如封面、中、英文题名页、独创性声明和使用授权书),或者双面打印只有1页的某部分内容(如摘要、ABSTRACT等)后插入1页空白页,该空白页不编排页眉页码;论文中出现的页码应前后连续,不得中断。


\section{本章小结}
本章介绍了……